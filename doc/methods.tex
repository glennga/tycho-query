\section{Methods}\label{sec:methods}
\subsection{Tycho2 Dataset}\label{subsec:tycho2Dataset}
The dataset used for all queries in this paper comes from the Tycho-2 dataset.


All experiments were performed on three node clusters, with each machine of type \texttt{n1-standard-1} (1 virtual CPU,
3.75GB of memory).
The dataset used is the ..

\subsection{Schema and Clustering}\label{subsec:schema}
Each star will have their Tycho IDs, current positions (right ascension, declination), proper motions, apparent
magnitude, and corresponding error terms saved as attributes.
The relationships between each star is their angular separation, only stored for stars that are within 10 degrees of
one another.

Each system will be allocated three nodes.

\subsubsection{Neo4J}
Representing the specified scheme above is natural if we treat each star as a node, each attribute as an attribute
of the node, and each relationship as an edge between each close star.

Loading each star was performed using the following statement for each node:
%\begin{flalign*}
%    &\texttt{CREATE (s:Star \{)} && \\
%    &\texttt{\hspace*{1cm} TYC1: \ldots} && \\
%    &\texttt{\hspace*{1cm} TYC1: \ldots} && \\
%    &\texttt{\hspace*{1cm} TYC1: \ldots} && \\
%    &\texttt{\hspace*{1cm} .} && \\
%    &\texttt{\hspace*{1cm} .} && \\
%    &\texttt{\hspace*{1cm} .} && \\
%    &\texttt{\})}
%\end{flalign*}

The relationships were generated as such (after loading each node):
%\begin{flalign*}
%    &\texttt{MATCH} &\texttt{(s1 : Star),(s2 : Star)} && \\
%    &\texttt{WHERE } &\theta(\texttt{s1, s2}) < \phi && \\
%    &\texttt{CREATE} &\texttt{(a)-[r : NEARBY \{} \theta(\texttt{s1, s2}) \texttt{\}]-(b)} && \\
%\end{flalign*}

The clustering used here will consist of one core server (for read and writes) and two read replicas.
Neo4J describes this as "casual clustering."

\subsubsection{Cassandra}
Representation of the scheme above involves two tables: one for the stars themselves, and another for the relationships
between each star.
The schema for the stars table is below:
%\begin{flalign*}
%    &\texttt{CREATE TABLE STARS (} && \\
%    &\texttt{\hspace*{1cm} TYC1 INT, } && \\
%    &\texttt{\hspace*{1cm} TYC2 INT, } && \\
%    &\texttt{\hspace*{1cm} TYC3 INT, } && \\
%    &\texttt{\hspace*{1cm} .} && \\
%    &\texttt{\hspace*{1cm} .} && \\
%    &\texttt{\hspace*{1cm} .} && \\
%    &\texttt{\hspace*{1cm} PRIMARY KEY(TYC1, TYC2, TYC2)} && \\
%    &\texttt{);} && \\
%
%    &\texttt{CREATE TABLE THETA (} && \\
%    &\texttt{\hspace*{1cm} TYC1_A INT, } && \\
%    &\texttt{\hspace*{1cm} TYC2_A INT, } && \\
%    &\texttt{\hspace*{1cm} TYC3_A INT, } && \\
%    &\texttt{\hspace*{1cm} TYC1_B} && \\
%    &\texttt{\hspace*{1cm} TYC2_B} && \\
%    &\texttt{\hspace*{1cm} TYC3_C} && \\
%    &\texttt{\hspace*{1cm} THETA} && \\
%    &\texttt{\hspace*{1cm} PRIMARY KEY(TYC1_A, TYC2_A, TYC3_A, TYC1_B, TYC2_B, TYC3_B)} && \\
%    &\texttt{);}
%\end{flalign*}






\subsection{Queries}\label{subsec:queries}
% A key search. Given the star identifier, which star

