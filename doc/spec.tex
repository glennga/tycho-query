\documentclass[11pt]{article}

\usepackage{a4wide} \usepackage{hyperref} \usepackage{listings, lstautogobble} \usepackage{xcolor} \usepackage{array}
\usepackage{amsmath}

\title{Benchmark Specification: Neo4J vs. Apache Cassandra for Astronomical Spatial Queries}
\author{Glenn Galvizo \\ University of Hawaii at Manoa} \date{\today}

\lstset{
    language=SQL,
    basicstyle=\scriptsize\ttfamily,
    commentstyle=\ttfamily\itshape\color{gray},
    stringstyle=\ttfamily,
    showstringspaces=false,
    breaklines=true,
    frameround=ffff,
%    frame=single,
%    rulecolor=\color{black},
    autogobble=true
}

\begin{document}
    \maketitle

    \section{Introduction}\label{sec:introduction}
    This benchmark is meant to characterize astronomical spatial queries on the column family implementation Apache
    Cassandra and the graph store implementation Neo4J\@.

    The repository that holds the paper and code associated with this benchmark can be found
    at~\url{https://github.com/glennga/tycho-query}.

    \section{Tycho-2 Dataset}\label{sec:tycho-2Dataset}
    This benchmark uses the Tycho-2 dataset, which can be found at~\url{ftp://cdsarc.u-strasbg.fr/pub/cats/I/259}.
    The bulk of the Tycho-2 star catalog comes from the ESA (European Space Agency) astrometric mission Hipparcos.
    There exists three different types of files of interest: \texttt{index.dat.gz}, \texttt{supp\_X.dat.gz}, and
    \texttt{tyc2.dat.XX.gz}.

    The first file represents the index file, and holds all GSC regions with their associated Tycho ID (\texttt{TYC1}),
    right ascension bounds, and declination bounds.
    The second file type represents the supplement files and holds all stars that could not be characterized
    reliably (due to brightness or unreliable measurements).
    This file holds the Tycho IDs of each star, their positions $(\alpha, \delta)$, their apparent magnitude, and
    the errors associated with each of the former.
    The third file type holds all of the stars that can be characterized reliably.
    In addition to the data recorded in the supplement files, these files also hold the epoch and error associated with
    the given measurement.

    This benchmark only uses the right ascension and declination associated data, and the apparent magnitude.

    \section{Neo4J Configuration}\label{sec:neo4jConfiguration}
    For the Neo4J DBMS, the files were loaded in order of supplement, regular, index.

    \subsection{Supplement and Regular Loading}\label{subsec:supplementAndRegularLoading}
    Each entry in the supplement and regular files were iterated through, and the following Cypher statements
    were executed.
    Entries with missing fields (i.e.\ non-stars) were not inserted.
    The items with the curly braces  (e.g. \texttt{\{ TYC1 \}}) represent the corresponding fields of that
    specific star.

    The following statements were executed once.
    \begin{lstlisting}
        // Create an index for star and regions nodes based on their TYC ID.
        CREATE INDEX ON :Star(TYC1, TYC2, TYC3);
        CREATE INDEX ON :Region(TYC1);
    \end{lstlisting}

    The following statements were executed for every star.
    \begin{lstlisting}
        // Insert the star node into the graph.
        CREATE (s:Star {
            TYC1: {TYC1},
            TYC2: {TYC2},
            TYC3: {TYC3},
            RAmdeg: {RAmdeg},
            DEmdeg: {DEmdeg},
            BTmag: {BTmag}
        });

        //  Create the region node in the graph if it does not already exist.
        MERGE (a:Region {
           TYC1: {TYC1}
        });

        // Create the appropriate relationship between the star and region node.
        MATCH (s:Star), (a:Region) WHERE
            s.TYC1 = {TYC1} AND
            s.TYC2 = {TYC2} AND
            s.TYC3 = {TYC3} AND
            a.TYC1 = {TYC1}
        CREATE (a)-[:CONTAINS]->(s);
    \end{lstlisting}

    \subsection{Index Loading}\label{subsec:indexLoading}
    Each entry in the index file was iterated through, and the following Cypher statement was executed.
    The \texttt{TYC1} property corresponds to its line number in the data file.

    This finds the region node created in~\autoref{subsec:supplementAndRegularLoading} and adds the appropriate
    right ascension and declination bounds as node properties:
    \begin{lstlisting}
        MATCH (a:Region {
            TYC1: {TYC1}
        }) SET a.RAmin = {RAmin}, a.RAmax = {RAmax}, a.DEmin = {DEmin}, a.DEmax = {DEmax};
    \end{lstlisting}

    Tests were performed with and without the indexes applied in~\autoref{subsec:supplementAndRegularLoading}.
    To delete these indexes, the following Cypher statements were executed.
    \begin{lstlisting}
        DROP INDEX ON :Star(TYC1, TYC2, TYC3);
        DROP INDEX ON :Region(TYC1);
    \end{lstlisting}

    \section{Cassandra Configuration}\label{sec:cassandraConfiguration}
    For the Apache Cassandra DBMS, the files were loaded in order of: index, supplement, regular.

    \subsection{Index Loading}\label{subsec:indexLoading2}
    Each entry in the index file was iterated through, and the following CQL statements were executed.
    The \texttt{TYC1} property corresponds to its line number in the data file.
    The \texttt{?} marks in the statements below indicate where the specific information of each region should be
    bound to.

    The following statements were executed once.
    \begin{lstlisting}
        ; Create the Tycho keyspace.
        CREATE KEYSPACE IF NOT EXISTS tycho
        WITH REPLICATION = {
            'class': 'SimpleStrategy',
            'replication_factor': '3'
        };

        ; Create the Region column family.
        CREATE TABLE IF NOT EXISTS tycho.region (
            TYC1 int,
            RAmin float,
            RAmax float,
            DEmin float,
            DEmax float,
            PRIMARY KEY (TYC1)
        );
    \end{lstlisting}

    The following statement was executed for every region.
    \begin{lstlisting}
        INSERT INTO tycho.region (
            TYC1, RAmin, RAmax, DEmin, DEmax
        ) VALUES (?, ?, ?, ?, ?);
    \end{lstlisting}

    \subsection{Supplement and Regular Loading}\label{subsec:supplementAndRegularLoading2}
    Each entry in the supplement and regular files were iterated through, and the following CQL statements
    were executed.
    Entries with missing fields (i.e.\ non-stars) were not inserted.

    The following statement was executed once.
    \begin{lstlisting}
        CREATE TABLE IF NOT EXISTS tycho.stars (
            TYC1 int,
            TYC2 int,
            TYC3 int,
            RAmdeg float,
            DEmdeg float,
            BTmag float,
            PRIMARY KEY (TYC1, TYC2, TYC3)
        );
    \end{lstlisting}

    The following statement was executed for each star in the data file.
    \begin{lstlisting}
        INSERT INTO tycho.stars (
            TYC1, TYC2, TYC3, RAmdeg, DEmdeg, BTmag
        ) VALUES (?, ?, ?, ?, ?, ?);
    \end{lstlisting}

    \section{Queries}\label{sec:queries}
    For each query listed in the following sections, the 22 stars in~\autoref{tab:starsInQueries} were used.
    One random star was selected from each data file to cover as much of the sky as possible.

    \section{Query 1}\label{sec:query1}
    The first query in this benchmark set asks what the characteristics are of some star $S$ given
    \texttt{TYC1-TYC2-TYC3}.

    \subsection{Cassandra Query 1}\label{subsec:cassandraQuery1}
    The following CQL was executed.
    \begin{lstlisting}
        SELECT *
        FROM tycho.stars
        WHERE TYC1 = ? AND TYC2 = ? AND TYC3 = ?;
    \end{lstlisting}

    \subsection{Neo4J Query 1}\label{subsec:neo4jQuery1}
    The following Cypher was executed.
    \begin{lstlisting}
        MATCH (s:Star)
        WHERE s.TYC1 = {TYC1} AND s.TYC2 = {TYC2} AND s.TYC3 = {TYC3}
        RETURN s;
    \end{lstlisting}

    \section{Query 2}\label{sec:query2}
    The second query asks what are the characteristics of stars near some star $s$ given $(\alpha, \delta)$.
    There exists two variants here: one that searches using the given position, and another searches for the
    \texttt{TYC1} ID after finding this in application logic.

    \subsection{Cassandra Query 2A}\label{subsec:cassandraQuery2a}
    Cassandra does not allow subqueries, so the results of the first query were fed into the results of the second.

    Search for the region given the position.
    \begin{lstlisting}
        SELECT TYC1
        FROM tycho.region
        WHERE RAmin < ? AND RAmax > ? AND DEmin < ? AND DEmax > ?
        ALLOW FILTERING;
    \end{lstlisting}

    Search for stars that share this region number.
    \begin{lstlisting}
        SELECT *
        FROM tycho.stars
        WHERE TYC1 = ?;
    \end{lstlisting}

    \subsection{Cassandra Query 2B}\label{subsec:cassandraQuery2b}
    The operation to find the \texttt{TYC1} ID was done outside of Cassandra.

    The following was performed in application logic.
    \begin{lstlisting}
        for each entry in the index file:
            if the current point is within the bounds:
                return TYC1
    \end{lstlisting}

    The results of the previous operation was then fed into the following CQL\@.
    \begin{lstlisting}
        SELECT *
        FROM tycho.stars
        WHERE TYC1 = ?
    \end{lstlisting}

    \subsection{Neo4J Query 2A}\label{subsec:neo4jQuery2a}
    The following Cypher was executed.
    \begin{lstlisting}
        MATCH (s:Star)-[:CONTAINS]-(a:Region)
        WHERE {RAmdeg} > a.RAmin AND {RAmdeg} < a.RAmax AND
              {DEmdeg} > a.DEmin AND {DEmdeg} < a.DEmax
        RETURN s;
    \end{lstlisting}

    \subsection{Neo4J Query 2B}\label{subsec:neo4jQuery2b}
    The following was performed in application logic.
    \begin{lstlisting}
        for each entry in the index file:
            if the current point is within the bounds:
                return TYC1
    \end{lstlisting}

    The results of the previous operation was then fed into the following Cypher.
    \begin{lstlisting}
        MATCH (s:Star)-[:CONTAINS]-(a:Region)
        WHERE a.TYC1 = {TYC1}
        RETURN s;
    \end{lstlisting}

    \section{Query 3}\label{sec:query3}
    The second query asks what are the characteristics of stars near some star $s$ given \texttt{TYC1-TYC2-TYC3}.
    There exists two variants for Neo4J: one that searches using only \texttt{TYC1} and one that searches for
    the specific star, traverses to the region node, and returns all stars associated with the given region.
    Cassandra can only perform the first variant.

    \subsection{Cassandra Query 3}\label{subsec:cassandraQuery3}
    The following CQL was executed.
    \begin{lstlisting}
        SELECT *
        FROM tycho.stars
        WHERE TYC1 = ?;
    \end{lstlisting}

    \subsection{Neo4J Query 3A}\label{subsec:neo4jQuery3a}
    The following Cypher was executed.
    \begin{lstlisting}
        MATCH (s:Star)
        WHERE s.TYC1 = {TYC1}
        RETURN s;
    \end{lstlisting}

    \subsection{Neo4J Query 3B}\label{subsec:neo4jQuery3b}
    The following Cypher was executed.
    \begin{lstlisting}
        MATCH (s:Star)-[:CONTAINS]-(a:Region), (z:Star)-[:CONTAINS]-(a:Region)
        WHERE s.TYC1 = {TYC1} AND s.TYC2 = {TYC2} AND s.TYC3 = {TYC3}
        RETURN s;
    \end{lstlisting}

    \section{Query 4}\label{sec:query4}
    The fourth query is identical to the second query with the inclusion of a brightness filter.

    \subsection{Cassandra Query 4A}\label{subsec:cassandraQuery4a}
    The following operations were performed.

    Search for the region given the position.
    \begin{lstlisting}
        SELECT TYC1
        FROM tycho.region
        WHERE RAmin < ? AND RAmax > ? AND DEmin < ? AND DEmax > ?
        ALLOW FILTERING;
    \end{lstlisting}

    Search for stars that share this region number and are visible with the naked eye.
    \begin{lstlisting}
        SELECT *
        FROM tycho.stars
        WHERE TYC1 = ? AND BTmag < 6.0;
    \end{lstlisting}

    \subsection{Cassandra Query 4B}\label{subsec:cassandraQuery4b}
    The operation to find the \texttt{TYC1} ID was done outside of Cassandra.

    The following was performed in application logic.
    \begin{lstlisting}
        for each entry in the index file:
            if the current point is within the bounds:
                return TYC1
    \end{lstlisting}

    The results of the previous operation was then fed into the following CQL\@.
    \begin{lstlisting}
        SELECT *
        FROM tycho.stars
        WHERE TYC1 = ? AND BTmag < 6.0;
    \end{lstlisting}

    \subsection{Neo4J Query 4A}\label{subsec:neo4jQuery4a}
    The following Cypher was executed.
    \begin{lstlisting}
        MATCH (s:Star)-[:CONTAINS]-(a:Region)
        WHERE {RAmdeg} > a.RAmin AND {RAmdeg} < a.RAmax AND
              {DEmdeg} > a.DEmin AND {DEmdeg} < a.DEmax AND s.BTmag < 6.0
        RETURN s;
    \end{lstlisting}

    \subsection{Neo4J Query 4B}\label{subsec:neo4jQuery4b}
    The following was performed in application logic.
    \begin{lstlisting}
        for each entry in the index file:
            if the current point is within the bounds:
                return TYC1
    \end{lstlisting}

    The results of the previous operation was then fed into the following Cypher.
    \begin{lstlisting}
        MATCH (s:Star)-[:CONTAINS]-(a:Region)
        WHERE a.TYC1 = {TYC1} AND s.BTmag < 6.0
        RETURN s;
    \end{lstlisting}

    \section{Query 5}\label{sec:query5}
    The fifth query is identical to thq third query with the inclusion of a brightness filter.

    \subsection{Cassandra Query 5}\label{subsec:cassandraQuery5}
    The following CQL was executed.
    \begin{lstlisting}
        SELECT *
        FROM tycho.stars
        WHERE TYC1 = ? AND BTmag < 6.0;
    \end{lstlisting}

    \subsection{Neo4J Query 5A}\label{subsec:neo4jQuery5a}
    The following Cypher was executed.
    \begin{lstlisting}
        MATCH (s:Star)
        WHERE s.TYC1 = {TYC1} AND s.BTmag < 6.0
        RETURN s;
    \end{lstlisting}

    \subsection{Neo4J Query 5B}\label{subsec:neo4jQuery5b}
    The following Cypher was executed.
    \begin{lstlisting}
        MATCH (s:Star)-[:CONTAINS]-(a:Region), (z:Star)-[:CONTAINS]-(a:Region)
        WHERE s.TYC1 = {TYC1} AND s.TYC2 = {TYC2} AND s.TYC3 = {TYC3} AND s.BTmag < 6.0
        RETURN s;
    \end{lstlisting}

    \newpage
    \section{Tables}\label{sec:tables}
        \begin{table}[h]
        \begin{center}
            \caption{
                List of all stars used in the query benchmark set.
            } \label{tab:starsInQueries}
            \begin{tabular}{ | m{0.3\linewidth} || m{0.3\linewidth} | m{0.3\linewidth} | }
                \hline
                \texttt{TYC} (catalog ID) & $\alpha$ (right ascension) & $\delta$ (declination) \\
                \hline \hline
                7018\ --- \ 01160\ --- \ 1 & 42.15935412 & -37.40235393 \\ \hline
                4532\ --- \ 02097\ --- \ 1 & 137.90154007 & 3.77865571 \\ \hline
                1034\ --- \ 00319\ --- \ 1 & 282.58271209 & 12.45930082 \\ \hline
                1148\ --- \ 01117\ --- \ 1 & 331.36198727 & 13.53828805 \\ \hline
                2112\ --- \ 00523\ --- \ 1 & 281.21247956 & 24.88268017 \\ \hline
                2471\ --- \ 01181\ --- \ 1 & 118.18754314 & 33.51838673 \\ \hline
                2936\ --- \ 00748\ --- \ 1 & 99.29161026 & 42.48758540 \\ \hline
                3386\ --- \ 00618\ --- \ 1 & 90.46745900 & 51.50930653 \\ \hline
                3875\ --- \ 01135\ --- \ 1 & 234.78965495 & 57.92442621 \\ \hline
                4227\ --- \ 00513\ --- \ 1 & 279.44191249 & 66.86263037 \\ \hline
                4896\ --- \ 00894\ --- \ 1 & 148.33662709 & -1.43938349 \\ \hline
                4901\ --- \ 00347\ --- \ 1 & 146.54433880 & -5.03752100 \\ \hline
                5980\ --- \ 01722\ --- \ 1 & 114.46513876 & -15.41866824 \\ \hline
                6554\ --- \ 02717\ --- \ 1 & 120.06655070 & -23.66481696 \\ \hline
                7120\ --- \ 01467\ --- \ 1 & 121.35783939 & -30.75622165 \\ \hline
                7640\ --- \ 01105\ --- \ 1 & 103.68753589 & -44.35031024 \\ \hline
                7899\ --- \ 00878\ --- \ 1 & 271.70230557 & -38.47822425 \\ \hline
                8348\ --- \ 00133\ --- \ 1 & 269.98544312 & -47.58312166 \\ \hline
                8717\ --- \ 02227\ --- \ 1 & 252.66602938 & -54.86935283 \\ \hline
                9004\ --- \ 02476\ --- \ 1 & 205.84510966 & -60.58699498 \\ \hline
                9537\ --- \ 00348\ --- \ 1 & 341.37590366 & -88.81829325 \\ \hline
            \end{tabular}
        \end{center}
    \end{table}

\end{document}