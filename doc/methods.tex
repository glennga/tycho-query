\section{Methods}\label{sec:methods}
The goal of this research was to characterize the running time of various queries under different data models, indices,
and architecture.

All experiments were performed on 1--3 node clusters, with each machine of type \texttt{n1-standard-1} (2 virtual CPUs,
3.75GB of memory).
To normalize the role of the cache (both OS and DBMS), each query set of the list in~\autoref{subsec:queries} was ran
45 times.
The last 15 runs were sampled and represents the data in~\autoref{sec:results}.
The repository that holds all code and data associated with this project can be found at:
\url{https://github.com/glennga/tycho-query}

\subsection{Queries}\label{subsec:queries}
Queries represent read operations on some DBMS\@.
Write operations were not tested here, as stars have not added to this dataset in years.
For simplicity, a star is defined as \textit{near} to another star if they share the same region.
A star is \textit{naked-eye visible} if its \texttt{BTmag} field (apparent magnitude) is below 6.0.

The following queries are to be examined:
\begin{enumerate}
    \item What are the characteristics of some star $S$?
    \item Which stars are near some Equatorial position $(\alpha, \delta)$?
    \item Which stars are near some star $S$, and what are their characteristics?
    \item Which stars are near some Equatorial position $(\alpha, \delta)$, and are naked-eye visible?
    \item Which stars are near some star $S$, and are naked-eye visible?
\end{enumerate}

Query 1 is a singular element search across the entire database.
Query 2 involves searching for some region that contains our point, and utilizing the data associated with the
region to search for stars.
This query can be approached two ways: computing the region ID given our point as application logic and searching with
this ID, or by searching with the bounds fields associated with each region.
Query 3 involves searching for all stars that share the same \texttt{TYC1} field.
Neo4J has the additional option of searching for the associated \textit{Region} node and walking each edge to the
corresponding \textit{Star} node.
Queries 4 and 5 are similar to 2 and 3 with an additional brightness filter.

Each general query was translated into 22 different queries for specific stars or locations.
When referencing the time to execute query 1 or 2, it is implied that this is the time to execute any of one the
specific queries for query set 1 or 2 on average.

\subsection{Cassandra}\label{subsec:cassandra}
The multi-node Cassandra clusters were configured with a \textit{Simple Snitch}, a replication factor = 3, and
\texttt{num\textunderscore tokens} = 256.
No other changes were made to the default configuration.
A snitch informs Cassandra about network topology for request efficiency, and the \textit{Simple Snitch}
is recommended for single-datacenter deployments.
Replication factor describes how many copies of the data exist, and \texttt{num\textunderscore tokens} determines
how much of the data each node gets.
Given that each node has identical hardware, the \texttt{num\textunderscore tokens} field indicates that all nodes
get the same amount of data.

All queries were operated serially and timed within the same session.
Cassandra does not allow subqueries (unlike SQL), so the time to execute the first query, feed this
into the result of the second, and execute this query was recorded as a single run.

For the first query in~\autoref{subsec:queries}, we are given the exact \texttt{TYC1-TYC2-TYC3} ID and are told to
search for all attributes associated with this star.
Given that our primary key involves this ID, all Cassandra has to do is run these numbers through the same hash function
used to partition the data, issue the query, and return the results.

The second query gives us some position instead of the \texttt{TYC} ID\@.
Our first approach (denoted as Query 2A) is to treat our column family as a relational table.
This involves searching for each "row" in our column family, and returning the \texttt{InRegion} field
if~\autoref{eq:isWithinRegion} holds:
\begin{equation}\label{eq:isWithinRegion}
    \left(\texttt{RAmin} < \alpha < \texttt{RAmax}\right) \land \left(\texttt{DEmin} < \delta < \texttt{DEmax}\right)
\end{equation}
From here, we search for all stars contained in the resultant of the previous query and return the result.

An alternative that avoids iterating through each primary key is to determine \texttt{TYC1} before asking Cassandra.
This is denoted as (Query 2B).
Query 2B involves iterating through a copy of the region data on the querying node, and checking for the region where
~\autoref{eq:isWithinRegion} holds.
From here, we perform the same secondary step as Query 2A\@.
A question of interest here is if this exchange for space results in a noticeable decrease in time.

Cassandra's third query takes advantage of our definition of near.
Here, we can just search for all stars sharing the same \texttt{TYC1}.
Queries 4 and 5 involve applying the naked-eye visibility filter to queries 2 and 3.
Another question of interest is how a \texttt{BTmag} index affects the speed of our query.

\subsection{Neo4J}\label{subsec:neo4j}
The multi-node Neo4J clusters were configured as \textit{Causal}, and all queries were performed on
the leader node.
No other changes were made to the default configuration.
All queries were executed serially, operating within the same session.

The first query in~\autoref{subsec:queries} is a search across the entire node list for a singular element given the
\texttt{TYC} ID\@.
This was tested with and without an index on \texttt{TYC1}, to determine how large the speedup is.

For the second query, we are given some position instead of the star ID itself.
The first approach, similar to Cassandra's 2A query, starts by searching for regions such
that~\autoref{eq:isWithinRegion} holds.
With the region known, we then traverse each connected \texttt{WITHIN} edge and return the properties of each
\textit{Star} node.

In line with Cassandra's 2B query, we also try to see if the same space exchange results in query speedup.
Neo4J's 2B query involves computing the \texttt{TYC1} ID in the same manner as Cassandra's query, but
then traversing the edges of the region node and returning the properties of each \textit{Star} node once found.
A question of interest here is how a \texttt{TYC1} index affects Neo4J's 2B query.

In the third query, we are tasked with finding the properties of stars near some star defined by its \texttt{TYC} ID\@.
The first approach (Query 3A) involves a similar query to Cassandra: search for all stars sharing the same
\texttt{TYC1} ID\@.
This is to be analyzed with and without indexes on \texttt{TYC1}.

The second approach (Query 3B) involves searching modeling the query in terms of relationships.
First, we search some star given the \texttt{TYC} ID\@.
Next, we traverse the connected \texttt{WITHIN} edge to find our region.
Finally, we traverse each \texttt{WITHIN} edge connected to our region and return the properties of each star.
Again, this was tested with and without a \texttt{TYC1} index.

Queries 4 and 5 are variants of queries 2 and 3 with the naked-eye visibility filter.
Similar to the Cassandra queries, a question of interest is how a \texttt{BTmag} index affects the query time.
