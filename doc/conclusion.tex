\section{Conclusion}\label{sec:conclusion}
\subsection{Future Work}\label{subsec:futureWork}
Future work for this project includes running more query variants, modifying each DBMS architecture more, and
analyzing the execution plan more.

A large assumption that helped simplify our queries was our definition of \textit{near}.
Sometimes, near might involve the angular separation between two stars that span over several GSC regions.
As our current data models stand, this would involve iterating through each star family / node set twice at worst.
The optimizations used here would not apply to this problem, and we would have to use an entirely different data
model.
 
Clusters of 1, 2, and 3 nodes at the same datacenter were used for each DBMS due to cost limitations.
With our current results, it seems that Neo4J should not scale out well and Cassandra should accept more nodes without
a problem.
Using more nodes across different regions would determine how well Neo4J and Cassandra really scale out.
Queries were also executed serially, and more interesting results may have been seen if multiple were run in parallel.
Both Neo4J and Cassandra provide availability, and characterizing this would also allow us to test each DBMS's
distributed nature.

Query plans were briefly looked at for the non-indexed Neo4J cluster running queries 1, 3A, and 5A\@.
Cassandra does provide an execution planner, and could be compared against Neo4J\@.
Query plans would provide more concrete insight into why certain queries run faster than others.
Delving deeper, more explicit analysis could be performed by characterizing the role of the cache for each query 
(cache misses or hits) or tracking the messages sent to each node. 

\subsection{Conclusion}\label{subsec:conclusion}
Finding nearby stars is a common stellar query: in star trackers, identifying poorly cataloged images, celestial
navigation\ldots
Unfortunately there are hundreds of millions recorded stars, and the problem becomes finding nearby stars in an
efficient manner.
Stars are recorded as 2D points on the celestial sphere in an astronomical catalog known as a star catalog.
Using the Tycho-2 star catalog allowed us to reduce this spatial query to finding stars that are contained
within a specific region on the sphere.

From here, we divided our approach in two: the Cassandra solution with \textit{Region} and \textit{Star} column
families, and the Neo4J solution with \textit{Region} nodes, \textit{Star} nodes, and \texttt{WITHIN} relationships.
The former is known for efficient, but restrictive queries.
The latter is known for its efficient relational queries, but also its inability to partition work like a column family.

Our results show that Cassandra runs faster when Neo4J is not indexed, but Neo4J is able to run just as efficiently
(in some cases, more) when the correct indexes are applied.
Cassandra queries benefit from taking the time to determine the primary key before asking Cassandra (time-space
tradeoff), instead of performing an exhaustive search on Cassandra itself.
Neo4J does not experiment this same performance boost.
Finally, Cassandra appears to perform the same under a change in cluster size while Neo4J does not.

For consistent performance, Cassandra should be selected.
For a more natural abstraction with a higher upper bound on performance, Neo4J should be selected.
